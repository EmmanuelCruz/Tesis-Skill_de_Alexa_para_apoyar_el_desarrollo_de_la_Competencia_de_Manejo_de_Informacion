%------------------------------------------------------------
%	Anexo 2
%------------------------------------------------------------

\chapter{Evaluación de la skill}
\label{Anexo2}

%------------------------------------------------------------
%	Cuestionario de entrada
%------------------------------------------------------------

\section{Cuestionario de entrada}
\label{B1Anexo}

\begin{tcolorbox}[colback=white!25!white,colframe=blue]
  \begin{enumerate}
    \item Edad
    \item Sexo
    \item ¿Qué semestre estás cursando?
    \item ¿Con qué frecuencia investigas en las siguientes fuentes? (Nunca, Casi nunca, Frecuentemente, Muy frecuentemente)
    \item ¿Qué dispositivos utilizas para realizar una investigación en Internet?
    \item Por lo general, ¿buscas temas de interés personal o con propósito académico en Internet?
    \item Al recuperar información (imagen, texto, video) ¿incluyes las referencias consultadas?
    \item Justifica tu respuesta anterior
    \item ¿Cómo consideras que es la información que obtienes al final de una investigación?
    \item De las siguientes fuentes de información, ¿cuáles sueles consultar?
    \item ¿Utilizas algún método de investigación?
    \item Si tu respuesta anterior fue afirmativa, ¿cuáles métodos de investigación has usado?
    \item ¿Conoces el modelo gavilán?
    \item Si tu respuesta anterior fue afirmativa, ¿alguna vez has llevado a la práctica el proceso de investigación con el modelo gavilán?
    \item ¿Cuánto tiempo te toma investigar un tema?
    \item ¿Sabes qué es un asistente basado en voz?
    \item Si tu respuesta anterior fue afirmativa, ¿cuál o cuáles asistentes basados en voz conoces?
    \item ¿Alguna vez has utilizado un asistente basado en voz?
    \item Si tu respuesta anterior fue afirmativa, ¿con qué fin sueles usar un asistente basado en voz?
    \item ¿Sabes qué es una skill para el asistente basado en voz de Amazon?
    \item Si has usado un asistente basado en voz. ¿Qué tan cómodo te sientes usándolo?
    \item ¿Crees que investigar con apoyo de un asistente basado en voz podría ayudar en el proceso de investigación?
    \item Justifica tu respuesta
  \end{enumerate}
\end{tcolorbox}

%------------------------------------------------------------
%	Protocolo de bienvenida
%------------------------------------------------------------

\section{Protocolo de bienvenida}
\label{B2Anexo}

\begin{tcolorbox}[colback=white!25!white,colframe=blue]
  Fecha de la prueba: \_\_\_\_ de \_\_\_\_\_\_\_\_\_\_\_\_\_\_\_ de 2021

  Buenos días/tardes, gracias por brindarnos unos minutos de su tiempo. Mi nombre es \_\_\_\_\_\_\_\_\_\_ y estaré contigo en esta sesión.

  Permíteme profundizar sobre la razón del porqué nos encontramos aquí.

  Estamos probando una skill que será un apoyo al proceso de investigación por medio del reconocimiento por voz. Antes de continuar, quiero comentarte que las skills son funcionalidades que permiten a los usuarios del asistente de voz Alexa de Amazon, crear una experiencia más personalizada.

  Esta sesión será grabada, ya que mediante el estudio de tus comentarios, podremos identificar las posibles mejoras de la skill, por lo que te pedimos que actúes con naturalidad. También te pedimos que expreses tus opiniones en voz alta, ya que este ejercicio tiene el propósito de evaluar a la skill y no a ti.

  Dado que es una skill que está en proceso de desarrollo, se podrían presentar situaciones imprevistas, así que te agradecemos considerarlo.

  Se presentará un prototipo y se te pedirá que realices algunas tareas típicas para las cuales está diseñada la skill. La sesión consistirá en que efectúes dichas tareas y describas en voz alta tus acciones, así como cualquier opinión que tengas, ya que serán de gran utilidad para mejorar la skill.

  Cuando comience la sesión, te puedes sentir en total libertad de hacer cualquier pregunta, aunque no podré contestar algunas de ellas, ya que el objetivo es simular una situación real en la que la skill opere de manera autónoma.
  Veamos un video sobre el funcionamiento de Alexa.

  Interactuarás con la Alexa que está con Emmanuel. Tu tarea será utilizar la skill llamada Buscador Gavilán, para investigar sobre un tema y realizar otras actividades que se permiten y se presentan a continuación.

  \begin{itemize}
    \item La skill se llama Buscador Gavilán
    \item Te permite investigar sobre un tema, para la prueba investigaras sobre matemáticas
    \item Te sugiere diferentes preguntas iniciales
    \item Te recomienda diferentes referencias
    \item Te da más información del tema
    \item Almacena la referencia de la información
    \item Puedes programar un recordatorio. Al final debes programar un recordatorio para el 10 diciembre.
  \end{itemize}

  Interactúa libremente con la skill, te pedimos que trates de probar todas las funciones. Puedes terminar cuando tú quieras.

  ¿Tienes alguna duda?
  Puedes comenzar a usar la skill.

\end{tcolorbox}

%------------------------------------------------------------
%	Actividades de la prueba
%------------------------------------------------------------

\section{Actividades de la prueba}
\label{B3Anexo}

\begin{tcolorbox}[colback=white!25!white,colframe=blue]
  Usuario No: \_\_\_\_\_\_\_\_\_\_\_\_\_\_\_\_\_\_\_\_\_\_\_\_\_\_\_\_\_\_\_\_\_\_\_\_\_\_\_\_\_\_\_\_\_\_\_\_\_\_\_\_\_\_
  Fecha de evaluación: \_\_\_\_\_\_\_\_\_ de \_\_\_\_\_\_\_\_\_\_\_\_\_\_\_\_\_\_\_\_\_\_\_ de 2021
  Hora de inicio: \_\_\_\_\_\_\_\_\_\_\_\_\_\_\_\_\_\_\_\_\_\_\_\_\_\_\_\_\_\_\_\_\_\_\_\_\_\_\_\_\_\_\_\_\_\_\_\_\_\_

  \textbf{Instrucciones para el moderador}
  \begin{enumerate}
    \item Ir tachando las actividades realizadas con el fin de no perderse durante la prueba. NO se realizará ninguna actividad si antes no ha concluido la anterior (a menos que se le instruya lo contrario).
    \item En caso de que el usuario cometa algún error preguntar:
    \textit{¿La skill le informó con claridad qué fue lo que pasó?}
  \end{enumerate}

  \begin{itemize}
    \item Activa la skill
    Nota para el monitor: Decir la frase “Alexa, abre buscador gavilán”
    \item Solicita a la skill investigar sobre el tema de matemáticas
    Nota para el monitor: Decir la frase “Alexa, quiero saber sobre matemáticas”
    \item Solicita a Alexa que te sugiera una nueva pregunta inicial
    Nota para el monitor: Decir la frase “Alexa, busca otra opción”
    \item Solicita a Alexa que utilice la pregunta inicial sugerida
    Nota para el monitor: Decir la siguiente frase “Alexa, sí es mi pregunta inicial”
    \item Solicita a Alexa que te recomiende otra referencia
    Nota para el monitor: Decir la siguiente frase “Alexa, consulta más fuentes”
    \item Solicita a Alexa que utilice la referencia propuesta
    Nota para el monitor: Decir la frase “Alexa, quiero esa referencia”
    \item Solicita a la skill más información del tema
    Nota para el monitor: Decir la frase “Alexa, lee siguiente párrafo”
    \item Solicita a la skill buscar otra referencia
    Nota para el monitor: Decir la frase ”Alexa, seguir investigando”
    \item Solicita a Alexa que almacene la referencia de la información
    Nota para el monitor: Decir la frase “Alexa, guarda la referencia”
    \item Solicita a Alexa que continúe con la investigación
    Nota para el monitor: Decir la frase “Alexa, continúa con la investigación”
    \item Solicita a Alexa finalizar con la investigación
    Nota para el monitor: Decir la frase “Alexa, termina la investigación”
    \item Solicita a la skill que programe un recordatorio para el 10 diciembre
    Nota para el monitor: Decir la frase “Alexa, agrega recordatorio para el 10 de diciembre”
    \item Solicita a Alexa salir de la skill
    Nota para el monitor: Decir la frase “Alexa, cierra la skill”
  \end{itemize}

  Gracias por participar en la evaluación.

  Anotar hora en que terminó la evaluación \_\_\_\_\_\_\_\_\_\_\_\_\_\_\_\_\_\_\_\_\_\_\_\_\_\_\_\_\_\_\_\_\_\_\_
  
  Notas del moderador:

  \_\_\_\_\_\_\_\_\_\_\_\_\_\_\_\_\_\_\_\_\_\_\_\_\_\_\_\_\_\_\_\_\_\_\_\_\_\_\_\_\_\_\_\_\_\_\_\_\_\_\_\_\_\_\_\_

  \_\_\_\_\_\_\_\_\_\_\_\_\_\_\_\_\_\_\_\_\_\_\_\_\_\_\_\_\_\_\_\_\_\_\_\_\_\_\_\_\_\_\_\_\_\_\_\_\_\_\_\_\_\_\_\_

  \_\_\_\_\_\_\_\_\_\_\_\_\_\_\_\_\_\_\_\_\_\_\_\_\_\_\_\_\_\_\_\_\_\_\_\_\_\_\_\_\_\_\_\_\_\_\_\_\_\_\_\_\_\_\_\_

  \_\_\_\_\_\_\_\_\_\_\_\_\_\_\_\_\_\_\_\_\_\_\_\_\_\_\_\_\_\_\_\_\_\_\_\_\_\_\_\_\_\_\_\_\_\_\_\_\_\_\_\_\_\_\_\_

  \_\_\_\_\_\_\_\_\_\_\_\_\_\_\_\_\_\_\_\_\_\_\_\_\_\_\_\_\_\_\_\_\_\_\_\_\_\_\_\_\_\_\_\_\_\_\_\_\_\_\_\_\_\_\_\_

  \_\_\_\_\_\_\_\_\_\_\_\_\_\_\_\_\_\_\_\_\_\_\_\_\_\_\_\_\_\_\_\_\_\_\_\_\_\_\_\_\_\_\_\_\_\_\_\_\_\_\_\_\_\_\_\_

\end{tcolorbox}

%------------------------------------------------------------
%	Guión de actividades para la evaluación
%------------------------------------------------------------

\section{Guión de actividades para la evaluación}
\label{B4Anexo}

\begin{tcolorbox}[colback=white!25!white,colframe=blue]
  \textbf{Funciones}
  \begin{itemize}
    \item La skill se llama Buscador Gavilán
    \item Te permite investigar sobre un tema, para la prueba investigaras sobre matemáticas
    \item Te sugiere diferentes preguntas iniciales
    \item Te recomienda diferentes referencias
    \item Te da más información del tema
    \item Almacena la referencia de la información
    \item Puedes programar un recordatorio, al final debes programar un recordatorio para el 10 diciembre
  \end{itemize}

  Interactúa libremente con la skill, te pedimos que trates de probar todas las funciones.
  Puedes terminar cuando tú quieras.

\end{tcolorbox}

%------------------------------------------------------------
%	Cuestionario de usabilidad
%------------------------------------------------------------

\section{Cuestionario de usabilidad}
\label{B5Anexo}

\begin{tcolorbox}[colback=white!25!white,colframe=blue]
  Lea cuidadosamente las siguientes afirmaciones y marque con una x que tan de acuerdo o desacuerdo está con ellas, considerando que 5 es equivalente a fuertemente desacuerdo y 1 a fuertemente de acuerdo. Si no está seguro (a) de que contestar marque 3.

  \begin{tabular}{| p{4cm} | p{2cm} | p{2cm} | p{2cm} | p{2cm} | p{2cm} |}
    \hline
     & \multicolumn{1}{p{2cm}}{Fuertemente en desacuerdo}  & \multicolumn{3}{p{2cm}}{} & Fuertemente de acuerdo \\ \hline
     & 5 & 4 & 3 & 2 & 1 \\ \hline
    Creo que me gustaría usar la skill frecuentemente. &  &  &  &  &  \\ \hline
    Encuentro la skill compleja. &  &  &  &  &  \\ \hline
    Pienso que la skill es fácil de usar. &  &  &  &  &  \\ \hline
    Creo que necesitaré la ayuda de un técnico para usar la skill. &  &  &  &  &  \\ \hline
    Encontré que las distintas funciones de la skill estaban bien integrados. &  &  &  &  &  \\ \hline
    Pienso que había mucha inconsistencia en la skill. &  &  &  &  &  \\ \hline
    Me imagino que la gente aprenderá a usar la skill bastante rápido. &  &  &  &  &  \\ \hline
    Encuentro la skill engorrosa de usar. &  &  &  &  &  \\ \hline
    Me sentí muy seguro usando la skill. &  &  &  &  &  \\ \hline
    Necesito aprender muchas cosas antes de poder utilizar la skill. &  &  &  &  &  \\ \hline
  \end{tabular}

\end{tcolorbox}

%------------------------------------------------------------
%	Cuestionario de percepción subjetiva
%------------------------------------------------------------

\section{Cuestionario de percepción subjetiva}
\label{B6Anexo}

\begin{tcolorbox}[colback=white!25!white,colframe=blue]
  \textbf{Reacciones globales de la skill}

  \begin{tabular}{| p{1cm} | p{1cm} | p{1cm} | p{1cm} | p{1cm} | p{1cm} | p{1cm} | p{1cm} | p{1cm} | p{1cm} |}
    \multicolumn{2}{p{1cm}}{Terrible} & \multicolumn{6}{p{1cm}}{} & \multicolumn{2}{p{1cm}}{Maravilloso} \\ \hline
    1 & 2 & 3 & 4 & 5 & 6 & 7 & 8 & 9 & NA \\ \hline
  \end{tabular}

  \begin{tabular}{| p{1cm} | p{1cm} | p{1cm} | p{1cm} | p{1cm} | p{1cm} | p{1cm} | p{1cm} | p{1cm} | p{1cm} |}
    \multicolumn{2}{p{1cm}}{Frustrante} & \multicolumn{6}{p{1cm}}{} & \multicolumn{2}{p{1cm}}{Satisfactoria} \\ \hline
    1 & 2 & 3 & 4 & 5 & 6 & 7 & 8 & 9 & NA \\ \hline
  \end{tabular}

  \begin{tabular}{| p{1cm} | p{1cm} | p{1cm} | p{1cm} | p{1cm} | p{1cm} | p{1cm} | p{1cm} | p{1cm} | p{1cm} |}
    \multicolumn{2}{p{1cm}}{Aburrida} & \multicolumn{6}{p{1cm}}{} & \multicolumn{2}{p{1cm}}{Estimulante} \\ \hline
    1 & 2 & 3 & 4 & 5 & 6 & 7 & 8 & 9 & NA \\ \hline
  \end{tabular}

  \begin{tabular}{| p{1cm} | p{1cm} | p{1cm} | p{1cm} | p{1cm} | p{1cm} | p{1cm} | p{1cm} | p{1cm} | p{1cm} |}
    \multicolumn{2}{p{1cm}}{Difícil} & \multicolumn{6}{p{1cm}}{} & \multicolumn{2}{p{1cm}}{Fácil} \\ \hline
    1 & 2 & 3 & 4 & 5 & 6 & 7 & 8 & 9 & NA \\ \hline
  \end{tabular}

  \begin{tabular}{| p{1cm} | p{1cm} | p{1cm} | p{1cm} | p{1cm} | p{1cm} | p{1cm} | p{1cm} | p{1cm} | p{1cm} |}
    \multicolumn{3}{p{2cm}}{Rígida} & \multicolumn{5}{p{1cm}}{} & \multicolumn{2}{p{1cm}}{Flexible} \\ \hline
    1 & 2 & 3 & 4 & 5 & 6 & 7 & 8 & 9 & NA \\ \hline
  \end{tabular}

  \hfill

  \textbf{Aprendizaje}

  Aprender a usar la skill

  \begin{tabular}{| p{1cm} | p{1cm} | p{1cm} | p{1cm} | p{1cm} | p{1cm} | p{1cm} | p{1cm} | p{1cm} | p{1cm} |}
    \multicolumn{2}{p{1cm}}{Difícil} & \multicolumn{6}{p{1cm}}{} & \multicolumn{2}{p{1cm}}{Fácil} \\ \hline
    1 & 2 & 3 & 4 & 5 & 6 & 7 & 8 & 9 & NA \\ \hline
  \end{tabular}

  Tiempo para aprender a usar la skill

  \begin{tabular}{| p{1cm} | p{1cm} | p{1cm} | p{1cm} | p{1cm} | p{1cm} | p{1cm} | p{1cm} | p{1cm} | p{1cm} |}
    \multicolumn{2}{p{1cm}}{Difícil} & \multicolumn{6}{p{1cm}}{} & \multicolumn{2}{p{1cm}}{Fácil} \\ \hline
    1 & 2 & 3 & 4 & 5 & 6 & 7 & 8 & 9 & NA \\ \hline
  \end{tabular}

  Recordar nombres y funcionalidades

  \begin{tabular}{| p{1cm} | p{1cm} | p{1cm} | p{1cm} | p{1cm} | p{1cm} | p{1cm} | p{1cm} | p{1cm} | p{1cm} |}
    \multicolumn{2}{p{1cm}}{Difícil} & \multicolumn{6}{p{1cm}}{} & \multicolumn{2}{p{1cm}}{Fácil} \\ \hline
    1 & 2 & 3 & 4 & 5 & 6 & 7 & 8 & 9 & NA \\ \hline
  \end{tabular}

  Las tareas se pueden realizar de manera directa

  \begin{tabular}{| p{1cm} | p{1cm} | p{1cm} | p{1cm} | p{1cm} | p{1cm} | p{1cm} | p{1cm} | p{1cm} | p{1cm} |}
    \multicolumn{2}{p{1cm}}{Difícil} & \multicolumn{6}{p{1cm}}{} & \multicolumn{2}{p{1cm}}{Fácil} \\ \hline
    1 & 2 & 3 & 4 & 5 & 6 & 7 & 8 & 9 & NA \\ \hline
  \end{tabular}

  Número de pasos por tarea

  \begin{tabular}{| p{1cm} | p{1cm} | p{1cm} | p{1cm} | p{1cm} | p{1cm} | p{1cm} | p{1cm} | p{1cm} | p{1cm} |}
    \multicolumn{2}{p{1cm}}{Difícil} & \multicolumn{6}{p{1cm}}{} & \multicolumn{2}{p{1cm}}{Fácil} \\ \hline
    1 & 2 & 3 & 4 & 5 & 6 & 7 & 8 & 9 & NA \\ \hline
  \end{tabular}

  Los pasos para completar una tarea siguen una secuencia lógica

  \begin{tabular}{| p{1cm} | p{1cm} | p{1cm} | p{1cm} | p{1cm} | p{1cm} | p{1cm} | p{1cm} | p{1cm} | p{1cm} |}
    \multicolumn{2}{p{1cm}}{Difícil} & \multicolumn{6}{p{1cm}}{} & \multicolumn{2}{p{1cm}}{Fácil} \\ \hline
    1 & 2 & 3 & 4 & 5 & 6 & 7 & 8 & 9 & NA \\ \hline
  \end{tabular}

  \hfill

  \textbf{Instrucciones de ayuda}

  La ayuda es

  \begin{tabular}{| p{1cm} | p{1cm} | p{1cm} | p{1cm} | p{1cm} | p{1cm} | p{1cm} | p{1cm} | p{1cm} | p{1cm} |}
    \multicolumn{2}{p{1cm}}{Confusa} & \multicolumn{6}{p{1cm}}{} & \multicolumn{2}{p{1cm}}{Clara} \\ \hline
    1 & 2 & 3 & 4 & 5 & 6 & 7 & 8 & 9 & NA \\ \hline
  \end{tabular}

  La información de la ayuda es fácilmente comprensible

  \begin{tabular}{| p{1cm} | p{1cm} | p{1cm} | p{1cm} | p{1cm} | p{1cm} | p{1cm} | p{1cm} | p{1cm} | p{1cm} |}
    \multicolumn{2}{p{1cm}}{Nunca} & \multicolumn{6}{p{1cm}}{} & \multicolumn{2}{p{1cm}}{Siempre} \\ \hline
    1 & 2 & 3 & 4 & 5 & 6 & 7 & 8 & 9 & NA \\ \hline
  \end{tabular}

  Cantidad de ayuda ofrecida

  \begin{tabular}{| p{1cm} | p{1cm} | p{1cm} | p{1cm} | p{1cm} | p{1cm} | p{1cm} | p{1cm} | p{1cm} | p{1cm} |}
    \multicolumn{2}{p{1cm}}{Inadecuada} & \multicolumn{6}{p{1cm}}{} & \multicolumn{2}{p{1cm}}{Adecuada} \\ \hline
    1 & 2 & 3 & 4 & 5 & 6 & 7 & 8 & 9 & NA \\ \hline
  \end{tabular}
\end{tcolorbox}

\begin{tcolorbox}[colback=white!25!white,colframe=blue]
  \textbf{Mensajes}

  Los mensajes que da Alexa me parecen

  \begin{tabular}{| p{1cm} | p{1cm} | p{1cm} | p{1cm} | p{1cm} | p{1cm} | p{1cm} | p{1cm} | p{1cm} | p{1cm} |}
    \multicolumn{2}{p{1cm}}{Confusos} & \multicolumn{6}{p{1cm}}{} & \multicolumn{2}{p{1cm}}{Claros} \\ \hline
    1 & 2 & 3 & 4 & 5 & 6 & 7 & 8 & 9 & NA \\ \hline
  \end{tabular}

  Las instrucciones sobre las actividades me parecen

  \begin{tabular}{| p{1cm} | p{1cm} | p{1cm} | p{1cm} | p{1cm} | p{1cm} | p{1cm} | p{1cm} | p{1cm} | p{1cm} |}
    \multicolumn{2}{p{1cm}}{Confusos} & \multicolumn{6}{p{1cm}}{} & \multicolumn{2}{p{1cm}}{Claros} \\ \hline
    1 & 2 & 3 & 4 & 5 & 6 & 7 & 8 & 9 & NA \\ \hline
  \end{tabular}

  Los mensajes de error me parecen

  \begin{tabular}{| p{1cm} | p{1cm} | p{1cm} | p{1cm} | p{1cm} | p{1cm} | p{1cm} | p{1cm} | p{1cm} | p{1cm} |}
    \multicolumn{2}{p{1cm}}{Confusos} & \multicolumn{6}{p{1cm}}{} & \multicolumn{2}{p{1cm}}{Claros} \\ \hline
    1 & 2 & 3 & 4 & 5 & 6 & 7 & 8 & 9 & NA \\ \hline
  \end{tabular}

  \hfill

  \textbf{Sobre la skill}

  Considero que la skill me ayudó a mejorar mi mecanismo de investigación

  \begin{tabular}{| p{1cm} | p{1cm} | p{1cm} | p{1cm} | p{1cm} | p{1cm} | p{1cm} | p{1cm} | p{1cm} | p{1cm} |}
    \multicolumn{2}{p{1cm}}{Poco} & \multicolumn{6}{p{1cm}}{} & \multicolumn{2}{p{1cm}}{Mucho} \\ \hline
    1 & 2 & 3 & 4 & 5 & 6 & 7 & 8 & 9 & NA \\ \hline
  \end{tabular}

  Considero que la skill me ayuda a seguir un proceso de investigación

  \begin{tabular}{| p{1cm} | p{1cm} | p{1cm} | p{1cm} | p{1cm} | p{1cm} | p{1cm} | p{1cm} | p{1cm} | p{1cm} |}
    \multicolumn{2}{p{1cm}}{Poco} & \multicolumn{6}{p{1cm}}{} & \multicolumn{2}{p{1cm}}{Mucho} \\ \hline
    1 & 2 & 3 & 4 & 5 & 6 & 7 & 8 & 9 & NA \\ \hline
  \end{tabular}

  La skill me parece innovadora

  \begin{tabular}{| p{1cm} | p{1cm} | p{1cm} | p{1cm} | p{1cm} | p{1cm} | p{1cm} | p{1cm} | p{1cm} | p{1cm} |}
    \multicolumn{2}{p{1cm}}{Poco} & \multicolumn{6}{p{1cm}}{} & \multicolumn{2}{p{1cm}}{Mucho} \\ \hline
    1 & 2 & 3 & 4 & 5 & 6 & 7 & 8 & 9 & NA \\ \hline
  \end{tabular}

  \hfill

  \textbf{La estrategia para investigar que usa la skill me parece}

  \begin{tabular}{| p{1cm} | p{1cm} | p{1cm} | p{1cm} | p{1cm} | p{1cm} | p{1cm} | p{1cm} | p{1cm} | p{1cm} |}
    \multicolumn{2}{p{1cm}}{Confusa} & \multicolumn{6}{p{1cm}}{} & \multicolumn{2}{p{1cm}}{Clara} \\ \hline
    1 & 2 & 3 & 4 & 5 & 6 & 7 & 8 & 9 & NA \\ \hline
  \end{tabular}

  \begin{tabular}{| p{1cm} | p{1cm} | p{1cm} | p{1cm} | p{1cm} | p{1cm} | p{1cm} | p{1cm} | p{1cm} | p{1cm} |}
    \multicolumn{2}{p{1cm}}{Muy lenta} & \multicolumn{5}{p{1cm}}{} & \multicolumn{3}{p{2cm}}{Suficientemente rápida} \\ \hline
    1 & 2 & 3 & 4 & 5 & 6 & 7 & 8 & 9 & NA \\ \hline
  \end{tabular}

  \begin{tabular}{| p{1cm} | p{1cm} | p{1cm} | p{1cm} | p{1cm} | p{1cm} | p{1cm} | p{1cm} | p{1cm} | p{1cm} |}
    \multicolumn{2}{p{1cm}}{Difícil} & \multicolumn{6}{p{1cm}}{} & \multicolumn{2}{p{1cm}}{Fácil} \\ \hline
    1 & 2 & 3 & 4 & 5 & 6 & 7 & 8 & 9 & NA \\ \hline
  \end{tabular}

  \hfill

  Comentarios:

  \_\_\_\_\_\_\_\_\_\_\_\_\_\_\_\_\_\_\_\_\_\_\_\_\_\_\_\_\_\_\_\_\_\_\_\_\_\_\_\_\_\_\_\_\_\_\_\_\_\_\_\_\_\_\_\_

  \_\_\_\_\_\_\_\_\_\_\_\_\_\_\_\_\_\_\_\_\_\_\_\_\_\_\_\_\_\_\_\_\_\_\_\_\_\_\_\_\_\_\_\_\_\_\_\_\_\_\_\_\_\_\_\_

  \_\_\_\_\_\_\_\_\_\_\_\_\_\_\_\_\_\_\_\_\_\_\_\_\_\_\_\_\_\_\_\_\_\_\_\_\_\_\_\_\_\_\_\_\_\_\_\_\_\_\_\_\_\_\_\_

  \_\_\_\_\_\_\_\_\_\_\_\_\_\_\_\_\_\_\_\_\_\_\_\_\_\_\_\_\_\_\_\_\_\_\_\_\_\_\_\_\_\_\_\_\_\_\_\_\_\_\_\_\_\_\_\_

  \_\_\_\_\_\_\_\_\_\_\_\_\_\_\_\_\_\_\_\_\_\_\_\_\_\_\_\_\_\_\_\_\_\_\_\_\_\_\_\_\_\_\_\_\_\_\_\_\_\_\_\_\_\_\_\_

  \_\_\_\_\_\_\_\_\_\_\_\_\_\_\_\_\_\_\_\_\_\_\_\_\_\_\_\_\_\_\_\_\_\_\_\_\_\_\_\_\_\_\_\_\_\_\_\_\_\_\_\_\_\_\_\_

\end{tcolorbox}
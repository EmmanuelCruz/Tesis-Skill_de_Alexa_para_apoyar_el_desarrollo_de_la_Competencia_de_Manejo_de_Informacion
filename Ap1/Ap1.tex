%------------------------------------------------------------
%	Anexo 1
%------------------------------------------------------------

\chapter{Código del Buscador Gavilán}
\label{Anexo1}

%------------------------------------------------------------
%	Web Scraper
%------------------------------------------------------------

\section{Web Scraper}
\label{A1Anexo}

\begin{tcolorbox}[colback=white!25!white,colframe=blue]
  \begin{minted}{python}
import requests
from bs4 import BeautifulSoup

class WebScraper:
  """
  Aplicador de Web Scraping para obtener el contenido de una 
  página web a partir de su URL.
  """

  def __init__(self, url):
    """
    Crea un nuevo web scraper
    Parámetro
    url -- url semilla de donde comenzar la búsqueda
    """
    self.url = url

    # Obtención del contenido de la url
    answer = requests.get(url)
    soup = BeautifulSoup(answer.text)

    # Lista de párrafos
    self.container_text = soup.find_all('p')

    # Contador de párrafo actual
    self.current = -1
    
  def has_next(self):
    """
    Verifica si quedan párrafos por leer.
    """
    return self.current+1 != len(self.container_text)
  \end{minted}
\end{tcolorbox}

\begin{tcolorbox}[colback=white!25!white,colframe=blue]
  \begin{minted}{python}
  def next(self):
    """
    Regresa el párrafo actual de lectura
    """
    self.current += 1
        
    while self.has_next() and len(self.container_text[self.current].text)<40:
      self.current += 1
        
    try:
      return self.container_text[self.current].text 
    except:
      return "Lo siento, no hay más párrafos por leer."
    
  def current_info(self):
    """
    Regresa el puntero al párrafo actual
    """
    if self.current != -1:
      return self.container_text[self.current].text
    else:
      return "Lo siento, no hay más párrafos por leer."
  \end{minted}
\end{tcolorbox}

Donde:

\begin{itemize}
  \item \textbf{Constructor.} La función que construye una nueva instancia de WebScraper, recibe como parámetro la url de la cuál se extraerán los datos. En el cuerpo de la función se crea una nueva instancia de BeautifulSoup y se obtienen todos los párrafos contenidos en el sitio.
  \item \textbf{has\_next.} Verifica si hay párrafos por recorrer en el sitio. Regresa \texttt{true} si existen párrafos y \texttt{false} en caso de haber recorrido todos los párrafos.
  \item \textbf{next.} Obtiene el párrafo seguido del párrafo actual en el recorrido.
  \item \textbf{current\_info.} Devuelve el párrafo actual del recorrido.
\end{itemize}

%------------------------------------------------------------
%	LaunchRequestHandler
%------------------------------------------------------------

\section{LaunchRequestHandler}
\label{A2Anexo}

\begin{tcolorbox}[colback=white!25!white,colframe=blue]
  \begin{minted}{python}
class LaunchRequestHandler(AbstractRequestHandler):
  """Handler for Skill Launch."""
  def can_handle(self, handler_input):
    # type: (HandlerInput) -> bool

    return ask_utils.is_request_type("LaunchRequest")(handler_input)

  def handle(self, handler_input):
    # type: (HandlerInput) -> Response
        
    speak_output = "Bienvenido al buscador gavilán. Te ayudaré a investigar"+ 
      "información de cualquier tema que desees. Puedes decir quiero saber sobre."+
      "Para comenzar con la investigación. ¿Sobre qué tema te gustaría investigar?"
    response_builder = handler_input.response_builder
        
    global help_id
    help_id = 0

    return response_builder.speak(speak_output).ask(speak_output).response
  \end{minted}
\end{tcolorbox}

%------------------------------------------------------------
%	CancelOrStopIntentHandler
%------------------------------------------------------------

\section{CancelOrStopIntentHandler}
\label{A3Anexo}

\begin{tcolorbox}[colback=white!25!white,colframe=blue]
  \begin{minted}{python}
class CancelOrStopIntentHandler(AbstractRequestHandler):
  """Single handler for Cancel and Stop Intent."""
  def can_handle(self, handler_input):
    # type: (HandlerInput) -> bool
    return (ask_utils.is_intent_name("AMAZON.CancelIntent")(handler_input) or
      ask_utils.is_intent_name("AMAZON.StopIntent")(handler_input))

  def handle(self, handler_input):
    # type: (HandlerInput) -> Response
    speak_output = "Gracias por investigar con la ayuda del buscador gavilán."+
      "Hasta pronto."

    return (
      handler_input.response_builder
        .speak(speak_output)
        .response
    )
  \end{minted}
\end{tcolorbox}

%------------------------------------------------------------
%	HelpIntentHandler
%------------------------------------------------------------

\section{HelpIntentHandler}
\label{A4Anexo}

\begin{tcolorbox}[colback=white!25!white,colframe=blue]
  \begin{minted}{python}
help_list = [
    "Puedes decir quiero saber sobre. Para comenzar una nueva investigación",
    "Puedes elegir una de las preguntas sugeridas o buscar otra pregunta "+
      "inicial diciendo sí o no",
    "Puedes elegir una de las referencias sugeridas o solicitar otra diciendo "+
      "sí o no",
    "Puedes solicitar guardar la referencia o pedir que diga el siguiente "+
      "párrafo si lo deseas",
    "Puedes solicitar más información del tema diciendo lee siguiente párrafo "+
      "o continúa con la investigación para consultar más referencias",
    "Puedes terminar la investigación o solicitar nuevas referencias",
    "Puedes decir agrega recordatorio u omitir recordatorio. Ya casi terminamos "+
      "la investigación",
    "Puedes decir cierra la skill para salir o preguntar sobre un nuevo tema"
]

class HelpIntentHandler(AbstractRequestHandler):
  """Handler for Help Intent."""
  def can_handle(self, handler_input):
    # type: (HandlerInput) -> bool
    return ask_utils.is_intent_name("AMAZON.HelpIntent")(handler_input)

  def handle(self, handler_input):
    # type: (HandlerInput) -> Response
    speak_output = help_list[help_id] + ". " + last_request

    return (
      handler_input.response_builder
        .speak(speak_output)
        .ask(speak_output)
        .response
      )
  \end{minted}
\end{tcolorbox}

%------------------------------------------------------------
%	FallbackIntentHandler
%------------------------------------------------------------

\section{FallbackIntentHandler}
\label{A5Anexo}

\begin{tcolorbox}[colback=white!25!white,colframe=blue]
  \begin{minted}{python}
class FallbackIntentHandler(AbstractRequestHandler):
  """Single handler for Fallback Intent."""
  def can_handle(self, handler_input):
    # type: (HandlerInput) -> bool
    return ask_utils.is_intent_name("AMAZON.FallbackIntent")(handler_input)

  def handle(self, handler_input):
    # type: (HandlerInput) -> Response
    logger.info("In FallbackIntentHandler")
    speech = "No entendí tu petición. Puedes decir ayuda para solicitar más "+
      "información. " + last_request
    reprompt = "No entendí tu petición. Puedes decir solicitar ayuda en "+
      "cualquier momento para apoyarte. " + last_request

    return handler_input.response_builder.speak(speech).ask(reprompt).response
  \end{minted}
\end{tcolorbox}

%------------------------------------------------------------
%	tellTopicIntentHandler
%------------------------------------------------------------

\section{tellTopicIntentHandler}
\label{A6Anexo}

\begin{tcolorbox}[colback=white!25!white,colframe=blue]
  \begin{minted}{python}
class tellTopicIntentHandler(AbstractRequestHandler):
  def can_handle(self, handler_input):
    # type: (HandlerInput) -> bool
    return ask_utils.is_intent_name("tellTopicIntent")(handler_input)

  def handle(self, handler_input):
        
    global questions
    global question
    global isQuestion
    global searcherContent
    global reference
    global mainTopic
    global list_info
    global last_request
    global list_titles
    global help_id
        
    questions = None
    help_id = 1
    question = 0
    isQuestion = True
    searcherContent = None
    reference = 0
    list_info = []
    last_request = ""
    list_titles = []
  \end{minted}
\end{tcolorbox}

\begin{tcolorbox}[colback=white!25!white,colframe=blue]
  \begin{minted}{python}    
    slotsFromIntent = handler_input.request_envelope.request.intent.slots
    topic = slotsFromIntent['searchWord']
    mainTopic = topic
    
    s = srch.Searcher(str(topic.value), True)
    
    questions = s.listQuestions()

    indice = random.randint(0, len(pregunta_inicial) - 1)
    speak_output = "Una pregunta inicial te ayudará a profundizar el tema que "+
      "deseas investigar, te presentaré algunas sugerencias para apoyarte." + 
      questions[question] + ". " + pregunta_inicial[indice]
    last_request = speak_output
    isQuestion = True

    return (
      handler_input.response_builder
        .speak(speak_output)
        .ask(speak_output)
        .response
      )
  \end{minted}
\end{tcolorbox}

%------------------------------------------------------------
%	yesAmazonIntentHandler
%------------------------------------------------------------

\section{yesAmazonIntentHandler}
\label{A7Anexo}

\begin{tcolorbox}[colback=white!25!white,colframe=blue]
  \begin{minted}{python}
class yesAmazonIntentHandler(AbstractRequestHandler):
  """Handler for Hello World Intent."""
  def can_handle(self, handler_input):
    # type: (HandlerInput) -> bool
    return ask_utils.is_intent_name("AMAZON.YesIntent")(handler_input)

  def handle(self, handler_input):
    # type: (HandlerInput) -> Response
    
    global searcherContent
    speak_output = ""
    global isQuestion
    global scraper
    global mainTopic
    global last_request
    global help_id
    
    last_request = ""
    
    if questions==None:
        speak_output = "Lo siento. Aún no tengo una pregunta inicial para iniciar"+
          " la investigación"
  \end{minted}
\end{tcolorbox}

\begin{tcolorbox}[colback=white!25!white,colframe=blue]
  \begin{minted}{python}   
    elif isQuestion:
      speak_output = "Has elegido la pregunta: " + questions[question]
      searcherContent = srch.Searcher(questions[question], False)
      searcherContent.lookForReferences()
      
      indice = random.randint(0, len(sugerencia_referencia)-1)
      speak_output += ". Encontré " + searcherContent.references[reference]['title']+ 
        " en " + searcherContent.references[reference]['displayLink'] + ". "
        +sugerencia_referencia[indice]
      help_id = 2
      isQuestion = False
    else:
      scraper = ws.WebScraper(searcherContent.references[reference]['link'],
        mainTopic)
      if scraper.has_next():
          nextInfo = str(scraper.next())
          speak_output = "Según " +
            searcherContent.references[reference]['displayLink']+
            " " + nextInfo + ", puedes decir 'guardar la referencia' para almacenar"+
            " o 'siguiente párrafo' para escuchar más de esta referencia o "+
            "'continuar con la investigación' para ir al siguiente paso"
          help_id = 3
          
      else:
          speak_output = "Lo siento, no pude acceder a la información de la"+
            " referencia. Aún así puedes almacenarla diciendo... guarda la "+
            "referencia"
  
    last_request = speak_output
        
    return (
      handler_input.response_builder
        .speak(speak_output)
        .ask(speak_output)
        .response
    )
  \end{minted}
\end{tcolorbox}

%------------------------------------------------------------
%	noAmazonIntentHandler
%------------------------------------------------------------

\section{noAmazonIntentHandler}
\label{A8Anexo}

\begin{tcolorbox}[colback=white!25!white,colframe=blue]
  \begin{minted}{python}
class noAmazonIntentHandler(AbstractRequestHandler):
  """Handler for Hello World Intent."""
  def can_handle(self, handler_input):
    # type: (HandlerInput) -> bool
    return ask_utils.is_intent_name("AMAZON.NoIntent")(handler_input)

  def handle(self, handler_input):
    # type: (HandlerInput) -> Response
    global question
    global questions
    global reference
    global last_request
    global help_id
    
    last_request = ""
        
    speak_output = ""
    if questions==None:
      speak_output = "Lo siento. Aún no tengo una pregunta inicial para iniciar la"+
      " investigación"
    elif isQuestion:
      question = (question + 1) % len(questions)
      speak_output = questions[question] + ". ¿Es tu pregunta inicial o busco otra "+
        "opción?"
      help_id = 1
    else:
      reference = (reference + 1) % len(searcherContent.references)
      speak_output = "Encontré " + searcherContent.references[reference]['title'] +
        " en " + searcherContent.references[reference]['displayLink'] + 
        ". ¿Consulto esta referencia o busco otra?"
      help_id = 2
    
    last_request = speak_output
    
    return (
      handler_input.response_builder
        .speak(speak_output)
        .ask(speak_output)
        .response
    )
  \end{minted}
\end{tcolorbox}

%------------------------------------------------------------
%	nextInfoIntentHandler
%------------------------------------------------------------

\section{nextInfoIntentHandler}
\label{A9Anexo}

\begin{tcolorbox}[colback=white!25!white,colframe=blue]
  \begin{minted}{python}
class nextInfoIntentHandler(AbstractRequestHandler):
  def can_handle(self, handler_input):
    # type: (HandlerInput) -> bool
    return ask_utils.is_intent_name("nextInfoIntent")(handler_input)

  def handle(self, handler_input):
        
    global scraper
    global last_request
    global help_id
        
    last_request = ""
    speak_output = ""
        
    if scraper == None:
      speak_output = "Lo siento, aún no has elegido tema o referencia"
    elif scraper.has_next():
      nextInfo = str(scraper.next())
      speak_output = nextInfo + ", puedes decir 'guardar la referencia' para "+
        "almacenar o 'siguiente párrafo' para escuchar más de esta referencia o "+
        "'continuar con la investigación' para ir al siguiente paso"
      help_id = 3
    else:
      nextInfoIntent = "Lo siento, ya no hay más información para mostrar en esta"+
        " referencia. Puedes almacenarla diciendo... Guarda la referencia"
      help_id = 3
    
    last_request = speak_output
    return (
      handler_input.response_builder
        .speak(speak_output)
        .ask(speak_output)
        .response
    )
  \end{minted}
\end{tcolorbox}

%------------------------------------------------------------
%	continueIntentHandler
%------------------------------------------------------------

\section{continueIntentHandler}
\label{A10Anexo}

\begin{tcolorbox}[colback=white!25!white,colframe=blue]
  \begin{minted}{python}
class continueIntentHandler(AbstractRequestHandler):
  def can_handle(self, handler_input):
    # type: (HandlerInput) -> bool
    return ask_utils.is_intent_name("continueIntent")(handler_input)

  def handle(self, handler_input):
        
    global last_request
    global help_id
        
    last_request = ""
        
    speak_output = "Puedes decir 'consultar más fuentes' para investigar más o"+
      " 'terminar la investigación' para continuar"
    help_id = 5
    last_request = speak_output
        
    return (
      handler_input.response_builder
        .speak(speak_output)
        .ask(speak_output)
        .response
    )
  \end{minted}
\end{tcolorbox}

%------------------------------------------------------------
%	endInvestigationIntentHandler
%------------------------------------------------------------

\section{endInvestigationIntentHandler}
\label{A11Anexo}

\begin{tcolorbox}[colback=white!25!white,colframe=blue]
  \begin{minted}{python}
class endInvestigationIntentHandler(AbstractRequestHandler):
  def can_handle(self, handler_input):
    # type: (HandlerInput) -> bool
    return ask_utils.is_intent_name("endInvestigationIntent")(handler_input)

  def handle(self, handler_input):
        
    global last_request
    global help_id
        
    help_id = 6
    last_request = ""
    speak_output = "No olvides agregar tu nombre y fuentes consultadas en la "+
      "investigación. Puedes decir 'agrega recordatorio' para no olvidar la fecha "+
      "de entrega. También puedes saltar el recordatorio diciendo 'omitir "+
      "recordatorio'"
    last_request = speak_output
  \end{minted}
\end{tcolorbox}

\begin{tcolorbox}[colback=white!25!white,colframe=blue]
  \begin{minted}{python}            
    return (
      handler_input.response_builder
        .speak(speak_output)
        .ask(speak_output)
        .response
    )
  \end{minted}
\end{tcolorbox}

%------------------------------------------------------------
%	addReminderIntentHandler
%------------------------------------------------------------

\section{addReminderIntentHandler}
\label{A12Anexo}

\begin{tcolorbox}[colback=white!25!white,colframe=blue]
  \begin{minted}{python}
class addReminderIntentHandler(AbstractRequestHandler):
  def can_handle(self, handler_input):
    # type: (HandlerInput) -> bool
    return ask_utils.is_intent_name("addReminderIntent")(handler_input)

  def handle(self, handler_input):
        
    global last_request
    global help_id
        
    last_request =""
    help_id = 7
        
    slotsFromIntent = handler_input.request_envelope.request.intent.slots
        
    date = slotsFromIntent['date'].value
    speak_output = "Tu recordatorio está listo para " + date
        
    request_envelope = handler_input.request_envelope
    response_builder = handler_input.response_builder
    reminder_service = handler_input.service_client_factory.
      get_reminder_management_service()
        
    # Permisos para recordatorios
    if not (request_envelope.context.system.user.permissions and
      request_envelope.context.system.user.permissions.consent_token):
            
      return response_builder.add_directive(
        SendRequestDirective(
          name="AskFor",
          payload={
            "@type": "AskForPermissionsConsentRequest",
            "@version": "1",
            "permissionScope": "alexa::alerts:reminders:skill:readwrite"
          },
          token="correlationToken"
        )
      ).response
  \end{minted}
\end{tcolorbox}

\begin{tcolorbox}[colback=white!25!white,colframe=blue]
  \begin{minted}{python}
    # Creación de recordatorio
    date_components = str(date).split("-")
    notification_time = date_components[0]+"-"+date_components[1]+"-"+
      date_components[2]+"T23:59:00"

    trigger = Trigger(object_type = TriggerType.SCHEDULED_ABSOLUTE, 
      scheduled_time = notification_time)
    text = SpokenText(locale='es-es', ssml = "<speak>Este es un recordatorio del "+
      "buscador Gavilán. No olvides entregar tu Investigación sobre" + 
      str(mainTopic.value) +"</speak>", text= 'Investigación de ' + 
      str(mainTopic.value) + ". No olvides "+"entregar tu Investigación.")
    alert_info = AlertInfo(AlertInfoSpokenInfo([text]))
    push_notification = PushNotification(PushNotificationStatus.ENABLED)
    reminder_request = ReminderRequest(notification_time , trigger, alert_info, 
      push_notification)

    try:
      reminder_response = reminder_service.create_reminder(reminder_request)
      logger.info("Reminder Created: {}".format(reminder_response))
    except ServiceException as e:
      logger.info("Exception encountered: {}".format(e.body))
      return response_builder.
        speak('Lo siento. No se pudo crear el recordatorio').response
        
    # Envío de datos al celular
    card_title = "Buscador Gavilán"
    card_text = ""
        
    if len(list_info) != 0:
      speak_output += ". Las referencias finales de la investigación son "
          
      for data in list_titles:
        speak_output += data + ". "
          
      message = ""
      for data in list_info:
        message += f"* {data}\n"
            
      speak_output += "Puedes consultar estas referencias en la app de Alexa. "
      #card_title = "Buscador Gavilán"
      card_text = message
  \end{minted}
\end{tcolorbox}

\begin{tcolorbox}[colback=white!25!white,colframe=blue]
  \begin{minted}{python} 
    remember_message = " Recuerda que puedes presentar la información en una "+
      "gráfica, un cuadro comparativo, una tabla, un artículo, un cuadro "+
      "sinóptico, un reporte o incluso en un dibujo. Gracias por investigar con"+
      " la ayuda del buscador gavilán, hasta pronto."
    speak_output += remember_message
        
    last_request = speak_output
        
    return (
      handler_input.response_builder
        .speak(speak_output)
        #.ask(speak_output) # Aquí termina la skill
        .response
    )
  \end{minted}
\end{tcolorbox}

%------------------------------------------------------------
%	omitIntentHandler
%------------------------------------------------------------

\section{omitIntentHandler}
\label{A13Anexo}

\begin{tcolorbox}[colback=white!25!white,colframe=blue]
  \begin{minted}{python}
class omitIntentHandler(AbstractRequestHandler):
  def can_handle(self, handler_input):
    # type: (HandlerInput) -> bool
    return ask_utils.is_intent_name("omitIntent")(handler_input)

  def handle(self, handler_input):
        
    global last_request
    global help_id
        
    help_id = 7
    last_request = ""
    speak_output = ""
        
    card_title = "Buscador Gavilán"
    card_text = ""
        
    if len(list_info) != 0:
      speak_output += "Las referencias finales de la investigación son las "+
        "siguientes. "
            
      for data in list_titles:
        speak_output += data + ". "
            
      message = ""
      for data in list_info:
        message += f"* {data}\n"
  \end{minted}
\end{tcolorbox}

\begin{tcolorbox}[colback=white!25!white,colframe=blue]
  \begin{minted}{python}
      speak_output += "Puedes consultar estas referencias en la app de Alexa."
      #card_title = "Buscador Gavilán"
      card_text = message
      
    remember_message = "Recuerda que puedes presentar la información en una "+
      "gráfica, un cuadro comparativo, una tabla, un artículo, un cuadro "+
      "sinóptico, un reporte o incluso en un dibujo. Gracias por investigar con "+
      "la ayuda del buscador gavilán, hasta pronto."
    speak_output += remember_message
         
    last_request = speak_output
        
    if len(list_info) != 0:
      return (
        handler_input.response_builder
          .speak(speak_output)
          .set_card(SimpleCard(card_title, card_text))
          #.ask(speak_output) # Aqui se termina la skill
          .response
      )
    else:
      return (
        handler_input.response_builder
          .speak(speak_output)
          #.ask(speak_output) # Aqui se termina la skill
          .response
      )
  \end{minted}
\end{tcolorbox}

%------------------------------------------------------------
%	lastRequestIntentHandler
%------------------------------------------------------------

\section{lastRequestIntentHandler}
\label{A14Anexo}

\begin{tcolorbox}[colback=white!25!white,colframe=blue]
  \begin{minted}{python}
class lastRequestIntentHandler(AbstractRequestHandler):
  def can_handle(self, handler_input):
    # type: (HandlerInput) -> bool
    return ask_utils.is_intent_name("lastRequestIntent")(handler_input)

  def handle(self, handler_input):
        
    global last_request
        
    speak_output = last_request
                
    return (
      handler_input.response_builder
        .speak(speak_output)
        .ask(speak_output)
        .response
    )

  \end{minted}
\end{tcolorbox}
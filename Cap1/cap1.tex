%------------------------------------------------------------
%	CAPITULO I
%------------------------------------------------------------

\chapter{Introducción}
\label{capI}


%------------------------------------------------------------
%	INTRODUCCIÓN
%------------------------------------------------------------

Las interfaces de usuario basadas en voz (Voice User Interfaces) permiten que los usuarios interactúen con el sistema a través de la voz. Actualmente, los ejemplos más conocidos de este tipo de tecnología son los asistentes basados en voz (voice assistant) tales como Siri, Google Assistant y Alexa. La interacción basada en voz ofrece diferentes ventajas entre ellas se pueden mencionar las siguientes: naturalidad y velocidad en la interacción, libertad de interacción sin uso de las manos y empatía.

En la educación los asistentes de voz ofrecen alternativas para personalizar el aprendizaje y el desarrollo de habilidades de los estudiantes. La introducción en la educación de esta tecnología ha crecido gracias a que las compañías que desarrollan estos asistentes, permiten la creación de aplicaciones y su uso sin costo.

Desde el punto de vista de la usabilidad, es un reto diseñar aplicaciones para estos asistentes, ya que deben permitir que los estudiantes sean capaces de utilizarlas sin la necesidad de la intervención de un profesor o tutor que los guíe para que puedan avanzar en el desarrollo de sus habilidades de forma autónoma.

Por otra parte, la pandemia derivada de la COVID-19 ha resaltado la necesidad de que los estudiantes desarrollen sus habilidades de búsqueda y organización de la información (information literacy), ya que los estudiantes tuvieron que recurrir a fuentes de información en línea y no siempre se revisaba que la información encontrada fuera válida, confiable y pertinente para resolver su problema de investigación. 

En este trabajo se explorará cómo se puede utilizar el asistente basado en voz Alexa para apoyar a los estudiantes a desarrollar sus habilidades de manejo de información, para ello se desarrollará una skill para Alexa que les ayude a llevar a cabo una investigación siguiendo el Modelo Gavilán. El Modelo Gavilán es una estrategia que guía a los estudiantes a resolver problemas de búsqueda de información.

La skill será diseñada para estudiantes de México de entre 15 y 21 años. Para establecer los requerimientos de la aplicación, se analizará la información que reportaron los participantes de un diplomado dirigido a profesores de bachillerato y licenciatura de México y Chile, sobre los problemas que presentan los estudiantes al realizar una investigación, como lo son las técnicas basadas en copy-paste, omitir el análisis de contenido y autores.

El objetivo de este trabajo es diseñar una skill de Alexa, basada en el diseño centrado en el usuario para apoyar a los estudiantes a mejorar el proceso de investigación de información con una metodología formal de competencias de información conocida como el Modelo Gavilán.

La comprobación de la hipótesis central, tiene como objetivo apoyar a los estudiantes a desarrollar sus habilidades de manejo de información, a través de una skill para Alexa que les ayude a llevar a cabo una investigación siguiendo el Modelo Gavilán.

La organización de este trabajo es la siguiente: el Modelo Gavilán, el diseño de interfaces de usuario basadas en voz, el desarrollo de la skill y la conclusión. En el capítulo 2, Modelo Gavilán, se desarrollan y ejemplifican los objetivos de cada paso y subpaso que compone el proceso de búsqueda de información definido por la metodología para el desarrollo de la Competencia de Manejo de Información (CMI) conocida como el Modelo Gavilán.

En el capítulo 3, diseño de interfaces de usuario basadas en voz, se desarrollan los conceptos más importantes de las interfaces de usuario, así como las características de las interfaces multimodales, en particular las interfaces de usuario basadas en reconocimiento por voz. Así mismo, se desarrollan los conceptos relacionados a la metodología de diseño centrado en el usuario, técnicas y herramientas para el análisis del usuario y documentos para realizar evaluaciones de un sistema con usuarios.

En el capítulo 4, desarrollo de la skill, se introducen los conceptos de la configuración y creación de skills en la consola de desarrollo de Alexa, tales como invocaciones, intenciones, declaraciones, slots, entre otros. Durante el desarrollo de la skill se presenta también el análisis del problema, en el que se identifican las principales dificultades que presentan los alumnos de entre 15 y 21 años durante el proceso de búsqueda de información, con lo cual se realiza una análisis del usuario, con el fin de diseñar el funcionamiento de la skill basada en la metodología de diseño centrado en el usuario. Posteriormente se presenta el desarrollo de la implementación de la skill, en la que se describe cada parte que compone el sistema y las técnicas aplicadas, tales como el Custom Search JSON API y el Web Scraping. Al final del capítulo se desarrollan y analizan los resultados obtenidos de la evaluación de la skill con usuarios, usando el cuestionario de usabilidad llamado System Usability Scale (SUS).

Finalmente, en el último capítulo se expondrán las conclusiones finales del trabajo, así como las oportunidades de trabajo a futuro para desarrollar módulos que mejoren la usabilidad y experiencia de usuario de la skill.

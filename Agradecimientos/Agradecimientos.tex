\chapter*{Agradecimientos}

Con mucho cariño, para mis padres Rosalía y Martiminano, mis hermanos Montserrat y Abraham, por todo el apoyo incondicional que me han brindado en mi formación personal, académica y profesional, que me ha permitido ser partícipe de numerosos logros y mucha satisfacción.

Agradezco profundamente a mi asesor, el Dr. Gustavo De la Cruz Martínez, por todo el apoyo, tiempo, guía, enseñanza y dedicación que me brindó durante clases y desarrollo de este trabajo. Realmente me orientó de maneras incontables para mi formación.

A mi mejor amigo Alexis, quien siempre me motivó a escribir esta tesis en los momentos alegres y complicados. Agradezco especialmente a mi mejor amiga Ailyn, quien ha estado conmigo desde el primer momento de la carrera, compartiendo risas, alegría, compañía, consejos y los recuerdos más gratos que tengo de la Facultad de Ciencias, por motivarme a continuar mis proyectos cuando se vieron sesgados por estrés, angustia o simplemente por malos momentos, por brindarme tu amistad incondional, gracias. Así mismo, agradezco a mis amigos que formaron parte importante de este trabajo y con quienes he compartido valiosos momentos, a Sara literalmente por todo, Luis Eduardo por ser el mejor equipo y compañero de clases, Karem por ser la clara representación de la amistad, Yessica Janeth por todos los momentos de confianza y motivación, Fhernanda por todas las charlas amenas y agradables.

A todos mis profesores, de quienes aprendí infinidad de conocimientos, que atesoro afectuosamente, y a su vez, hicieron crecer mi pasión por las Ciencias de la Computación, en particular a Elisa Viso Gurovich, Pedro Ulises Cervantes González, Carlos Zerón Martínez, Selene Marisol Martínez Ramírez y Manuel Ignacio Castillo López.

A mis alumnos, que amablemente participaron durante las evaluaciones del proyecto, por tomarse un momento para ayudarme a implementar mejoras significativas. Finalmente, agradezco a los sinodales y a todos aquellos que se tomen el tiempo de leer esta tesis que disfruté mucho escribiendo.

Estos agradecimientos, son un escrito eficaz contra el olvido, que honran a quien honor merece, por la confianza, la virtud, el tiempo, el apoyo y la ayuda que me han ofrecido para convertirme en quien soy ahora. Lejos de agradecer por compromiso o educación, agradezco sinceramente por cada cambio revelador, no sólo de esta tesis, sino también de mi vida.
